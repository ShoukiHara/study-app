\documentclass[dvipdfmx]{jsarticle}
\usepackage[T1]{fontenc}
\usepackage{lmodern}
\usepackage[top=25mm, bottom=25mm, left=25mm, right=25mm]{geometry}
\usepackage{color}
\usepackage{listings}
\usepackage{tcolorbox}
\tcbuselibrary{breakable, skins}

% 見出し番号を消す設定
\setcounter{secnumdepth}{0}

% コードブロックのスタイル設定
\lstset{
    basicstyle=\ttfamily\small,
    backgroundcolor=\color[gray]{0.95},
    frame=single,
    breaklines=true,
    columns=fullflexible,
    keepspaces=true,
    xleftmargin=0mm,
    xrightmargin=0mm
}

% 用語解説用のボックス設定
\newtcolorbox{termbox}[1]{
    colback=green!5!white,
    colframe=green!55!black,
    title=\textbf{#1},
    breakable,
    enhanced,
    width=\linewidth
}

% 重要な概念用のボックス
\newtcolorbox{conceptbox}[1]{
    colback=orange!5!white,
    colframe=orange!65!black,
    title=\textbf{★ #1},
    breakable,
    enhanced,
    width=\linewidth
}

% ★追加:これが抜けていたためエラーになっていました
\newtcolorbox{warnbox}[1]{
    colback=red!5!white,
    colframe=red!75!black,
    title=\textbf{【注意】 #1},
    breakable,
    enhanced,
    width=\linewidth
}

\title{\textbf{Webアプリ開発の基礎:バックエンド完全ガイド}}
\author{開発学習ノート}
\date{\today}

\begin{document}

\maketitle
\tableofcontents
\newpage

\section{はじめに}
本ドキュメントは、PythonとFastAPIを用いたバックエンド(サーバーサイド)開発の仕組みと、各ファイルの役割を整理したものである。「厨房」の役割を果たすサーバーが、どのように動いているかを解説する。

\section{第1部:バックエンド開発の流れ}

\subsection{Step 1: 環境の隔離(Virtual Environment)}
Python開発では、プロジェクトごとに「専用の道具箱」を作るのが鉄則である。

\begin{termbox}{venv(ブイ・エンブ)}
「Virtual Environment(仮想環境)」の略。パソコン全体の設定を汚さずに、このアプリ専用の Python 環境を作る機能。
\end{termbox}
\begin{itemize}
    \item コマンド: \texttt{python -m venv venv}
    \item 有効化: \texttt{.{\textbackslash}venv{\textbackslash}Scripts{\textbackslash}Activate} (これを行うと行頭に \texttt{(venv)} が付く)
\end{itemize}

\subsection{Step 2: サーバー構築(FastAPI)}
\begin{termbox}{FastAPI(ファスト・エーピーアイ)}
PythonでWeb API(アプリの窓口)を作るための最新フレームワーク。高速で、コードが書きやすく、自動でドキュメントを作ってくれる優れもの。
\end{termbox}

\begin{termbox}{Uvicorn(ユービコーン)}
書いたPythonコードを、実際にWebサーバーとして動かすための実行ソフト。FastAPIとセットで使う。
\end{termbox}

\subsection{Step 3: 通信の許可(CORS)}
\begin{conceptbox}{CORS(コアーズ:Cross-Origin Resource Sharing)}
Webブラウザのセキュリティ機能。「違う場所(オリジン)からのアクセス」をブロックする仕組み。\par
\textbf{今回の状況:}
\begin{itemize}
    \item フロントエンド: \texttt{localhost:5173} (React)
    \item バックエンド: \texttt{localhost:8000} (Python)
\end{itemize}
住所(ポート番号)が違うため、何もしないと通信がブロックされる。バックエンド側で「5173からのアクセスはOKだよ」と許可証(Allow Origins)を発行する必要がある。
\end{conceptbox}

\newpage

\section{第2部:ファイル構成図解(Backend Map)}
\texttt{study-api} フォルダ内の構成。

\subsection{1. 実行ファイル}
\begin{itemize}
    \item \textbf{\texttt{main.py}} \\
    \textbf{【役割:厨房の司令塔】} \\
    バックエンドの全てのロジックが書かれたファイル。
    \begin{itemize}
        \item \textbf{CORS設定}: Reactからの接続を許可する設定。
        \item \textbf{BaseModel (Pydantic)}: 「どんなデータ(型)を受け取るか」を定義した伝票。今回は「学年」「教科」「勉強時間」などを定義した。
        \item \textbf{@app.post("/diagnose")}: 実際の注文受付窓口。データを受け取り、処理(今回はモックの返信)をして返す場所。
    \end{itemize}
\end{itemize}

\subsection{2. 環境・システムファイル}
\begin{itemize}
    \item \textbf{\texttt{venv/} (フォルダ)} \\
    \textbf{【役割:専用の道具箱】} \\
    このフォルダの中に、インストールしたライブラリ(FastAPIなど)の実体が入っている。
    \begin{itemize}
        \item \textbf{注意}: このフォルダはGitなどにアップロードしてはいけない(サイズが大きいため)。
    \end{itemize}
\end{itemize}

\section{第3部:フロントエンドとの連携の仕組み}

\subsection{1. リクエスト(注文)}
React(フロントエンド)の \texttt{fetch} 関数が、Pythonサーバーの \texttt{/diagnose} というURLにデータを投げる。
\begin{lstlisting}[language=java]
// フロントエンド側のコードイメージ
fetch('http://127.0.0.1:8000/diagnose', {
    method: 'POST',
    body: JSON.stringify(formData) // 入力データをJSONにして送る
})
\end{lstlisting}

\subsection{2. レスポンス(料理の提供)}
Python(バックエンド)がデータを受け取り、辞書型(dict)で返事を返す。FastAPIが自動的にこれをJSON(Webの標準データ形式)に変換してReactに届ける。

\begin{lstlisting}[language=python]
# バックエンド側のコードイメージ
return {
    "summary": "AIからのアドバイス...",
    "books": [...]
}
\end{lstlisting}

\section{開発中の注意点}
\begin{warnbox}{2つのターミナルを維持する}
アプリを動かすためには、以下の2つが同時に動いている必要がある。
\begin{enumerate}
    \item \textbf{React (Wait Staff)}: \texttt{npm run dev}
    \item \textbf{FastAPI (Chef)}: \texttt{uvicorn main:app --reload}
\end{enumerate}
片方でも止まっていると、画面が表示されなかったり、診断ボタンを押してもエラーになったりする。
\end{warnbox}

\end{document}
